\documentclass[a4paper,12p]{article}
\usepackage{standalone}
\usepackage{polski}
\usepackage[utf8]{inputenc}
\usepackage{blindtext}


\begin{document}
\begin{titlepage}
   \vspace*{\stretch{1.0}}
   \begin{center}
 	\Huge\textbf{Dokumentacja} 
	\bigbreak
      \Large\textbf{Analiza ilości terenów zielonych na terenie dzielnic Warszawy.}\\
      \large\textit{Bartosz Jasiński, Marta Rutkowska} \\	
	\bigbreak
	\bigbreak
	\Large\textbf{v 1.0} \\
	\small{Warszawa 24.04.2018}
	
   \end{center}
   \vspace*{\stretch{2.0}}
\end{titlepage}

	\tableofcontents

	\section{Wstęp}
	\subsection{Realizowany cel biznesowy}
		Analiza stosunku pola przestrzeni zielonych do ilości mieszkańców, procentu zazielenia i gęstości zaludnienia dzielnic Warszawy na podstawie zdjęć satelitarnych oraz danych administracyjnych i demograficznych Warszawy. 
	\subsection{Użytkowanie aplikacji}
		Aplikacja wyświetla mapę Warszawy podzieloną na dzielnice, z nałożoną na nią heatmapą wskazującą na obszary zielone. Użytkownik ma możliwosć wybrania z mapy interesującej go dzielnicy. Zostanie mu zwrócona informacja o gęstosci zaludnienia, ilosci powierzchini terenów zielonych przypadających w niej na mieszkańca oraz jaki procent obszaru danej dzielnicy jest terenem zielonym.
	\subsection{Technologia}
		Projekt wykonany w technologii: Python po stronie serwera najprawdopodobniej z użyciem frameworku tornado, JavaScrip oraz HTML po stronie przeglądarki

	
	\section{Dane aplikacji}
	\subsection{Źródła danych}
		Dane użyte w projekcie pochodzą z publicznie udostępnianych, darmowych źródeł wymienionych poniżej.
	
	\begin{itemize}
	\item Zdjęcia satelitarne z Google maps api: https://maps.googleapis.com
	\item Dane na temat podziału miasta Warszawa na dzielnice pozyskane ze strony http://www.gugik.gov.pl, pełny link do danych: http://www.gugik.gov.pl/geodezja-i-kartografia/pzgik/dane-bez-oplat/dane-z-panstwowego-rejestru-granic-i-powierzchni-jednostek-podzialow-terytorialnych-kraju-prg
	\item Dane na temat ludności dzielnic Warszawy, stan na 2017 rok z portalu http://stat.gov.pl, pełny link do danych: http://stat.gov.pl/obszary-tematyczne/ludnosc/ludnosc/powierzchnia-i-ludnosc-w-przekroju-terytorialnym-w-2017-r-,7,14.html
	\end{itemize}

	\subsection{Baza danych aplikacji}
	Dane na temat ludnosci poszczególnych dzielnic Warszawy przechowywane są w pliku csv. Podział miasta na dzielnice wczytywany jest z pliku Shapefile.

	\section{Praca z aplikacją}
		\subsection{Uruchomienie}

		\subsection{Obsługa aplikacji}

	\section{Utrzymanie aplikacji}
		Utrzymanie aplikacji będzie opierało się głównie o odswieżanie danych na temat ludnosći dzielnic Warszawy oraz ewentualnym odswieżeniu danych o granicach administracyjnych dzielnic w razie ich zmiany, lub utworzenia nowych jednostek.


\end{document}