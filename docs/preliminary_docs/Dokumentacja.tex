\documentclass[a4paper,12p]{article}
\usepackage{standalone}
\usepackage{polski}
\usepackage[utf8]{inputenc}

\begin{document}

	\tableofcontents

	\section{Wstęp}
	\subsection{Realizowany cel biznesowy}
		Analiza stosunku przestrzeni zielonych do ilości mieszkańców na podstawie zdjęć satelitarnych oraz danych demograficznych Warszawy.
	\subsection{Użytkowanie aplikacji}
		Aplikacja wyświetla mapę Warszawy podzieloną na jej dzielnice. Użytkownik ma możliwosć wybrania interesującej go dzielnicy. Zostanie mu zwrócona informacja o ilosci powierzchini terenów zielonych przypadających na głowę mieszkańca danej dzielnicy.
	\subsection{Technologia}
		Projekt wykonany w technologii: Python po stronie serwera najprawdopodobniej z użyciem frameworku tornado, JavaScrip oraz HTML po stronie przeglądarki

	
	\section{Źródła danych aplikacji}
		Dane użyte w projekcie pochodzą z publicznie udostępnianych, darmowych źródeł wymienionych poniżej.
	
	\begin{itemize}
	\item Zdjęcia satelitarne z Google maps api: https://maps.googleapis.com
	\item Dane na temat podziału miasta Warszawa na dzielnice pozyskane ze strony http://www.gugik.gov.pl, pełny link do danych: http://www.gugik.gov.pl/geodezja-i-kartografia/pzgik/dane-bez-oplat/dane-z-panstwowego-rejestru-granic-i-powierzchni-jednostek-podzialow-terytorialnych-kraju-prg
	\item Dane na temat ludności dzielnic Warszawy, stan na 2017 rok z portalu http://stat.gov.pl, pełny link do danych: http://stat.gov.pl/obszary-tematyczne/ludnosc/ludnosc/powierzchnia-i-ludnosc-w-przekroju-terytorialnym-w-2017-r-,7,14.html
	\end{itemize}

\end{document}