\documentclass[11pt]{article}
\usepackage{graphicx}
\usepackage[T1]{fontenc}
\usepackage[polish]{babel}
\usepackage{tabularx}
\usepackage[table,xcdraw]{xcolor}
\usepackage[utf8]{inputenc}
\usepackage{lmodern}
\usepackage{multirow}
\usepackage{array}
\usepackage{booktabs}
\selectlanguage{polish}
\usepackage{titlesec}
\usepackage{amsmath}
\usepackage{esint}
\usepackage{textpos}
\usepackage{chngpage}
\usepackage{calc}
\usepackage{placeins}
\usepackage{longtable}
\usepackage{hyperref}

\setlength\parindent{24pt}

\begin{document}

	
\begin{titlepage}				
\centering

{\Large{Wydział Matematyki i Nauk Informacyjnych Politechniki Warszawskiej}}

\vspace{1cm}		
{\includegraphics[scale=0.25]{logo}}
\vspace{3cm}

\Huge\textbf{Dokumentacja} 
\bigbreak

{\Huge Analiza ilości terenów zielonych na terenie dzielnic Warszawy}
\vspace{1cm}

{\Large Bartosz Jasiński, Marta Rutkowska}
\vspace{1cm}

\Large{v1.0}
\vspace{1cm}

{\itshape {\Large \today}}

\vfill		

\end{titlepage}


\tableofcontents


\newpage

	\section{Wstęp}
	\subsection{Realizowany cel biznesowy}
		Analiza stosunku pola przestrzeni zielonych do ilości mieszkańców, procentu zazielenia i gęstości zaludnienia dzielnic Warszawy na podstawie zdjęć satelitarnych oraz danych administracyjnych i demograficznych Warszawy.
		
	\subsection{Użytkowanie aplikacji}
		Aplikacja wyświetla mapę Warszawy podzieloną na dzielnice na które będzie nałożony pół-przejrzysty kolor (od czerwonego do zielonego) w zależności od procentu powierzchni zajmowanej przez tereny zielone w danej dzielnicy. Użytkownik ma możliwosć wybrania z mapy interesującej go dzielnicy. Podane zostaną następujące informacje:
    \begin{itemize}
      \item gęstości zaludnienia
      \item liczby mieszkańców  
      \item ilości powierzchni terenów zielonych przypadających w danym obszarze na mieszkańca
      \item jaki procent powierzchni danego obszaru zajmują obszary zielone
    \end{itemize}
Zapytania będą zwracać wcześniej wyliczone wartości przetrzymywane na serwerze (w czasie klikania na daną dzielnicę). 


	\subsection{Technologia}
		Projekt wykonany w technologii: \textbf{Python} po stronie serwera najprawdopodobniej z użyciem frameworku \textbf{Tornado}, razem z bibliotekami pomocnymi w obsłudze plików \textbf{csv}, \textbf{JavaScrip} oraz \textbf{HTML} po stronie przeglądarki.  

	
	\section{Dane aplikacji}
	\subsection{Źródła danych}
		Dane użyte w projekcie pochodzą z publicznie udostępnianych, darmowych źródeł wymienionych poniżej.
	
	\begin{itemize}
	\item Zdjęcia satelitarne z Google Maps API: https://maps.googleapis.com
	\item Dane na temat podziału miasta Warszawa na dzielnice pozyskane z serwisu WFS http://wfs.um.warszawa.pl/serwis
	\item Dane na temat ludności dzielnic Warszawy, stan na 2017 rok z portalu http://stat.gov.pl, pełny link do danych: http://stat.gov.pl/obszary-tematyczne/ludnosc/ludnosc/powierzchnia-i-ludnosc-w-przekroju-terytorialnym-w-2017-r-,7,14.html
	\end{itemize}

	\section{Przetworzenie danych w aplikacji}

	\subsection{Dane na temat granic dzielnic Warszawy}
	Ze względu na awaryjność serwisu serwującego dane na temat podziału miasta Warszawa na dzielnice w formacie WFS oraz dużej niezmienności tych danych zdecydowaliśmy się na pobranie danych i korzystanie z nich w aplikacji w sposób statyczny. W celu przekonwertowania ich do odpowiedniego systemu EPSG napisaliśmy własny konwerter tych danych otrzymując plik z danymi gotowymi do użycia w aplikacji.

	\subsection{Dane na temat ludności dzielnic Warszawy}
	Dane na powyższy temat uzyskaliśmy ze strony \url{https://warszawa.stat.gov.pl/files/gfx/warszawa/pl/defaultstronaopisowa/1484/1/1/monitoring_ludnosc_201506_waw_t1.pdf} są one z dnia 1.01.2015.

	\subsection{Wyliczenie procenta obszaru zielonego na zdjęciu satelitarnym}
	Do detekcji terenów zielonych skorzystaliśmy z biblioteki OpenCV (https://opencv.org) na licencji BSD (open source). 

	Algorytm działa w następujący sposób:
	\begin{itemize}
	\item Wybrany obraz jest wpierw ładowany w programie i konwertowany z RGB do HSV
	\item Tworzona jest maska przepuszczająca tylko kolory z pewnego wybranego zakresu (w tej aplikacji wszystkie odcienie koloru zielonego), oraz ustawiana jest pewna wrażliwość maski (margines błędu)
	\item Obraz pierwotny poddany jest maskowaniu (funkcja biblioteki OpenCV) i powstaje obraz wynikowy, na którym znajdują się jedynie piksele o kolorach dopuszczonych przez maskę
	\item Wyliczany jest procent nie pustych pikseli na wynikowym obrazie w stosunku do ilości pikseli w obrazie początkowym i wyświetlany jest użytkownikowi wynik
	\end{itemize}

	Efekty działania algorytmu wyznaczania terenów zielonych widzimy na poniższym obrazku. Obszary które nie zostały uznane przez nasz algorytm jako zielone na wynikowym obrazku prezentują się jako czarne obszary.
	\includegraphics[scale=0.3]{green_areas_algorithm} 

	\subsection{Baza danych aplikacji}
	Dane na temat ludnosci poszczególnych dzielnic Warszawy przechowywane są w pliku csv po stronie serwera. Dane na temat granic poszczególnych dzielnic były początkowo zapisane zgodnie z formatem EPSG:2178, przekonwertowaliśmy te dane na format EPSG:3857 ponieważ OpenLayers wymaga współrzędnych w takim formacie. Te dane trzymamy po stronie serwera i wysyłamy do klienta w celu ich narysowania.

\section{Architektura aplikacji}
	\subsection{Klient}
	Klientem w naszej aplikacji jest przeglądarka której jest serwowana strona internetowa z mapą Warszawy z zaznaczonymi granicami dzielnic. Klient po otrzymaniu eventu kliknięcia na mapę, wysyła request do serwera który wyszukuje w którą klikneliśmy dzielnicę (na podstawie współrzędnych kliknięcia), po czym zwraca odpowiednie dane do klienta który wyświetla je dla użytkownika
	\subsection{Serwer}
	Serwer został napisany w Pythonie przy użyciu modułu \textbf{socketserver}, głównym zadaniem serwera jest serwowanie strony internetowej z mapą dla przeglądarki oraz zwracanie danych na temat dzielnic w odpowiedzi na requesty od klienta. Serwer podczas przetważania odpowiedzi sprawdza 
	
	\section{Utrzymanie aplikacji}
		Utrzymanie aplikacji będzie opierało się głównie o odswieżanie danych na temat ludnosći dzielnic Warszawy oraz ewentualnym odswieżeniu danych o granicach administracyjnych dzielnic w razie ich zmiany, lub utworzenia nowych jednostek oraz uaktualniania zdjęć satelitarnych które służą do szacowania ilości terenów zielonych.
		
		
		\section{Obsługa aplikacji}
		Użytkowanie naszej aplikacji sporowadza się do odwiedzenia strony internetowej, na której wyświetlana jest mapa Warszawy z podziałem na dzielnice. Użytkownik ma możliwość kliknięcia na daną dzielnicę po czym zostaną mu wyświetlone dane na jej temat. Główny ekran aplikacji widoczny jest na poniższym obrazku
		
\includegraphics[scale=0.3]{app_main_screen} 


\end{document}
